\newpage

\section*{Annex A: Validation for SSH Isolation \& Fail2ban Hardening}  
\label{annexa}  

\begin{lstlisting}[language=bash, label={annexa:network}, caption={Network Isolation Verification}]  
# Verify port 22 redirects to Cowrie (2222) and admin port (2223) is exclusive  
sudo nmap -sV -p 22,2222,2223 $EC2_PUBLIC_IP  

# Check iptables NAT rules for redirect (should show 22 to 2222)  
sudo iptables -t nat -L PREROUTING -v -n  

# Ensure no SSH service binds to port 22 (only Cowrie on 2222)  
sudo ss -tulpn | grep -E ':22|:2222|:2223'  
\end{lstlisting}  

\begin{lstlisting}[language=bash, label={annexa:fail2ban}, caption={Fail2ban Efficacy Testing}]  
# Simulate key-based brute-forcing to trigger fail2ban  
for i in {1..5}; do ssh -i /path/fake_key ubuntu@localhost -p 2223; done  

# Verify fail2ban logged the bans (look for 'ssh-admin' jail)  
sudo grep "ssh-admin" /var/log/fail2ban.log  

# Check active bans (should list test IP)  
sudo fail2ban-client status ssh-admin  
\end{lstlisting}  

\begin{lstlisting}[language=bash, label={annexa:cowrie}, caption={Cowrie Logging Validation}]  
# Replay attacker commands via SSH  
echo 'echo "malicious payload"' | ssh attacker@localhost -p 22  

# Verify Cowrie logged the activity (JSON format)  
sudo jq '.eventid, .input' /var/log/cowrie/cowrie.json | tail -n 20  

# Monitor fake file tampering (e.g., /etc/fake_shadow)  
sudo auditctl -w /etc/fake_shadow -p war -k cowrie_tampering  
\end{lstlisting}  



\newpage

