\newpage

\newpage 


\section{Annex: LLM Usage in this Project}  
\label{annex:llm}  
\
\

Large Language Models (LLMs) were strategically employed during the realization of this project, mainly as \emph{collaborative augmentation tools} rather than content generators. The following guidelines must be respected in order to keep control over a LLM work.
\subsection*{Methodological Approach}

--- \textbf{Verification-Centric Deployment}: LLMs output is heavily dependent on its user's own comprehension, AI will not do more than instructed to, it will not \texttt{consciously} ask user a question, it will not compel you to do anything, but it is a decent corrector and will definitely be able to give suggestions that go in the right direction. All model outputs undergo cross-validation against credible and up-to-date sources such as; official documentation (Linux man pages, git repository documentation, Cloud guides), academic literature (Spitzner's foundational honeypot frameworks, cybersecurity courses), security best practices and standards. Not one technical implementation (e.g.: IAM policies, iptables rules, or authentication logic) were deployed before thorough validation and edge-case mitigation.

--- \textbf{Controlled Application}: LLMs interactions were constrained to: debugging assistance for Python scripting and \textit{systemd}-based Linux administration, technical documentation refinement (post-human draft), architectural brainstorming (especially in understanding distinct public Cloud providers' lingo for equivalent functionalities). Original research material or system designs created solely from LLMs output alone were absolutely forbidden and so was plagiarism. To conclude, results of this research are human-produced only or consequences of attack patterns that were witnessed within our honeypot network.

\subsection*{Operational Safeguards}
The implementation incorporated safeguards to preserve academic integrity:

--- \textbf{Input Sanitization Protocol}: All project-specific identifiers (IP addresses, credentials, API
keys) were redacted automatically before any LLM interaction, preventing potential data leakage through model training vectors.


--- \textbf{Bias Mitigation}: From a model's output to the distinction of attackers' behavioral patterns, all information gathered during execution was triangulated with honeypot telemetry data and peer-reviewed between our group to counter potential algorithmic hallucinations. Model-derived hypotheses of attacker tactics were utilized solely as preliminary filters before our own analysis.

\



\newpage

\section{Annex: Validation for SSH Isolation \& Fail2ban Hardening} 
\label{annex:network}
\
\
\begin{lstlisting}[language=bash, label={annexes:network}, caption={Network Isolation Verification}]  
# Port 22 redirects to Cowrie (2222) and admin port (61001) is exclusive  
sudo nmap -sV -Pn -p 22,2222,610001 51.79.248.60

# Check iptables NAT rules for redirect
sudo iptables -t nat -L PREROUTING -v -n  
\end{lstlisting}  
\


\begin{lstlisting}[language=bash, label={annexes:fail2ban}, caption={Fail2ban Efficacy Testing}]  
# Simulate brute-forcing administrative access 
for i in {1..5}; do ssh -i ~/.ssh/wrong_key.pem ubuntu@51.79.248.60 -p 61001; done  

# Check active bans  
sudo fail2ban-client status ssh-admin  
\end{lstlisting}  
\

\begin{figure}[h!]
\centering
    \includegraphics[width=0.8\linewidth]{doc/img/annex_b_listing_2.png}
\end{figure}

\begin{lstlisting}[language=bash, label={annexes:ssh-hardening}, caption={SSH Service Hardening Validation}]  
# 1. SSH configuration resists downgrade attempts
sudo sshd -T | grep -E '^ciphers|^kexalgorithms|^macs|^hostkeyalgorithms'  


# 3. Confirm password authentication is disabled 
ssh -o PubkeyAuthentication=no -o PreferredAuthentications=password ubuntu@51.79.248.60 -p 61001  
\end{lstlisting}

\begin{figure}[h!]
    \centering
    \includegraphics[width=0.8\linewidth]{doc/img/annex_b_listing_3.png}
\end{figure}
\





\newpage

\section{Annex: Cowrie Operational Validation}  
\label{annex:cowrie-validation}  
\
\
\begin{lstlisting}[language=bash,label={lst:cowrie-redirect},caption={Traffic Redirection Verification}]  
# 1. Validate iptables NAT rules  
sudo iptables -t nat -L PREROUTING -nv | grep 'tcp dpt:22 redir ports 2222'  

# Confirm no direct binding to port 22  
sudo nmap -sV -Pn -p 22,2222 51.79.248.60 | grep -E '22/tcp|2222/tcp'  
\end{lstlisting}  
\
\begin{lstlisting}[language=bash,label={lst:cowrie-access},caption={Honeypot Engagement Testing}]  
# 2. Simulate attacker connection  
ssh -o StrictHostKeyChecking=no invalid_user@51.79.248.60 -p 22  

# Inspect Honeypot interactions in real time  
sudo apt-get install ccze
sudo tail -f /home/cowrie/cowrie/var/log/cowrie/cowrie.log | ccze -A  
\end{lstlisting}  
\begin{figure}[h!]
    \centering
    \includegraphics[width=1\linewidth]{doc/img/annex_c_listing_2.png}
    \caption*{Record of two intrusions in real time}
\end{figure}

\


\
% END