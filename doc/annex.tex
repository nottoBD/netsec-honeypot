\newpage
% \section*{Annexes}

\section{Annex: Validation for SSH Isolation \& Fail2ban Hardening}  
\label{annex:network}
\
\
\begin{lstlisting}[language=bash, label={annexes:network}, caption={Network Isolation Verification}]  
# Verify port 22 redirects to Cowrie (2222) and admin port (2223) is exclusive  
sudo nmap -sV -Pn -p 22,2222,2223 $EC2_PUBLIC_IP  

# Check iptables NAT rules for redirect (should show 22 to 2222)  
sudo iptables -t nat -L PREROUTING -v -n  

# Ensure no SSH service binds to port 22 (only Cowrie on 2222)  
sudo ss -tulpn | grep -E ':22|:2222|:2223'  
\end{lstlisting}  
\


\begin{lstlisting}[language=bash, label={annexes:fail2ban}, caption={Fail2ban Efficacy Testing}]  
# Simulate key-based brute-forcing to trigger fail2ban  
for i in {1..5}; do ssh -i ~/.ssh/wrong_key.pem ubuntu@localhost -p 2223; done  

# Verify fail2ban logged the bans (look for 'ssh-admin' jail)  
sudo grep "ssh-admin" /var/log/fail2ban.log  

# Check active bans (should list test IP)  
sudo fail2ban-client status ssh-admin  
\end{lstlisting}  
\


\begin{lstlisting}[language=bash, label={annexes:ssh-hardening}, caption={SSH Service Hardening Validation}]  
# 1. Verify active SSH configuration matches hardening intent (no fallback to weak protocols)  
sudo sshd -T | grep -E '^ciphers|^kexalgorithms|^macs|^hostkeyalgorithms'  

# 2. Test SSH service for protocol/cipher negotiation weaknesses  
nmap -Pn -p 2223 --script ssh2-enum-algos $EC2_PUBLIC_IP | grep -A 10 "algorithm negotiation"  

# 3. Confirm password authentication is globally disabled (even if bypass attempted)  
ssh -o PubkeyAuthentication=no -o PreferredAuthentications=password ubuntu@$EC2_PUBLIC_IP -p 2223  
\end{lstlisting}  
\





\newpage

\section{Annex: Cowrie Operational Validation}  
\label{annex:cowrie-validation}  
\
\
\begin{lstlisting}[language=bash,label={lst:cowrie-redirect},caption={Traffic Redirection Verification}]  
# 1. Validate iptables NAT rules  
sudo iptables -t nat -L PREROUTING -nv | grep 'tcp dpt:22 redir ports 2222'  

# Confirm no direct binding to port 22  
sudo nmap -sV -Pn -p 22,2222 $EC2_IP | grep -E '22/tcp|2222/tcp'  
\end{lstlisting}  
\
\begin{lstlisting}[language=bash,label={lst:cowrie-access},caption={Honeypot Engagement Testing}]  
# 2. Simulate attacker connection  
ssh -o StrictHostKeyChecking=no invalid_user@$EC2_IP -p 22  

# Verify session capture in logs  
sudo journalctl -u cowrie -f | grep 'SSH connection closed'  
\end{lstlisting}  
\
\begin{lstlisting}[language=bash,label={lst:cowrie-context},caption={Process Isolation Validation}]  
# 3. Confirm execution context  
ps -ef | grep cowrie | grep -v grep | awk '{print $1}' | uniq  
\end{lstlisting}  
\



\newpage

% NEW ANNEX HERE

\newpage 


\section{Annex: LLM Usage in this Project}  
\label{annex:llm}  
\
\
Large language models (LLMs) provided targeted support during this honeypot deployment, strictly limited to non-operational tasks. For documentation, LLMs assisted in drafting initial LaTeX templates for technical sections such as SSH hardening (2.1) and iptables redirection (2.3), with all configurations manually validated against AWS and Cowrie documentation. During troubleshooting, models proposed diagnostic commands for SSH permission conflicts (e.g., 'chmod' adjustments in §3.2), which were later tested in isolated Docker environments before EC2 implementation. LLMs were explicitly excluded from security-critical decisions: firewall rules, Fail2ban thresholds, and cryptographic settings in '/etc/ssh/sshd\_config' derived exclusively from NIST guidelines and Mozilla Infosec recommendations. No model influenced attacker engagement strategies, log analysis of captured TTPs, or live system interactions. 
All AI-generated content underwent peer review by the project’s cybersecurity team, with particular scrutiny applied to network redirection mechanics and user permission workflows. Final configurations reflect human expertise, with LLMs serving solely as productivity accelerators for non-sensitive administrative tasks.    

% END