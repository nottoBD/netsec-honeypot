%main.tex
\documentclass{cls/ULBreport}

% Fix section numbering (remove chapter prefix)
\renewcommand{\thesection}{\arabic{section}}
\renewcommand{\thesubsection}{\thesection.\arabic{subsection}}
\sceau{img/sceauULB.jpg}
\addbibresource{bib/biblio.bib}

% Keep fourier fonts for the title
\usepackage{ragged2e}
\usepackage{parskip}
\geometry{a4paper,top=1.5cm,bottom=2cm,left=2cm,right=2cm,headheight=15pt,includehead}

% For \AfterGroup hook
\usepackage{etoolbox} 

% Configure abstract (compact, italic, bold, centered with "Abstract" title)
\renewenvironment{abstract}{%
  \clearpage
  \thispagestyle{plain}
  \begin{center}
    \bfseries\itshape ABSTRACT
  \end{center}
  \vspace{0.5cm}
  \begin{quote}
    \itshape
    \justifying
}{%
  \end{quote}
  \clearpage
}

% Add listings config (unchanged)
\lstset{
    basicstyle=\ttfamily\small,
    breaklines=true,
    captionpos=b,
    frame=single,
    numbers=left,
    numberstyle=\tiny\color{gray},
}
\hypersetup{
    hidelinks,
    colorlinks=true,
    linkcolor=blue,
    urlcolor=blue
}

% Customize citation commands for footnotes
\DeclareFieldFormat{footnote:note}{#1}
\DeclareFieldFormat{footnote:shorttitle}{\mkbibemph{#1}}
\DeclareFieldFormat{footnote:author}{#1}
\DeclareFieldFormat{footnote:year}{\mkbibparens{#1}}
\DeclareFieldFormat{footnote:url}{\url{#1}}
% Adjust footnote spacing if needed
\setlength{\footnotesep}{0.5em}

% Customization for the bibliography
\DeclareFieldFormat{url}{\url{#1}}
\DeclareFieldFormat{note}{\textit{#1}}
\addbibresource{refs/references.bib}

% Custom format for code
\lstset{
    basicstyle=\ttfamily\small,
    breaklines=true,
    captionpos=b,
    frame=single,
    numbers=left,
    numberstyle=\tiny\color{gray},
}
% Define vibrant colors
\definecolor{background}{rgb}{1.0,1.0,1.0} % White background
\definecolor{keyword}{rgb}{0.1,0.1,0.8}    % Deep blue for keywords
\definecolor{comment}{rgb}{0.5,0.5,0.5}    % Gray for comments
\definecolor{string}{rgb}{0.8,0.1,0.1}     % Strong red for strings
\definecolor{identifier}{rgb}{0.0,0.0,0.5} % Navy blue for identifiers and functions
\definecolor{number}{rgb}{0.6,0.2,0.2}     % Reddish-brown for numbers

% Configure listings package
\lstset{
    language=C,
    backgroundcolor=\color{background},
    basicstyle=\ttfamily\footnotesize,    % Small monospace font
    keywordstyle=\color{keyword}\bfseries, % Keywords in bold blue
    commentstyle=\color{comment}\itshape, % Comments in italic gray
    stringstyle=\color{string},           % Strings in strong red
    identifierstyle=\color{identifier},   % Identifiers in navy blue
    numberstyle=\tiny\color{comment},     % Line numbers in gray
    numbers=left,                         % Line numbers on the left
    stepnumber=1,                         % Line number increment
    numbersep=8pt,                        % Line number spacing
    frame=single,                         % Single-line frame around code
    rulecolor=\color{black},              % Frame color
    breaklines=true,                      % Enable line breaking
    showstringspaces=false,               % Don't show spaces in strings
    tabsize=4,                            % Set tab size
    captionpos=b,                         % Caption position at the bottom
    morekeywords={uint8_t, uint16_t, uint32_t, uint64_t} % Add custom keywords if needed
}


\begin{document}

    \titleULB{
    	title={Deception \& Honeypot for Attack Profiling},
    	studies={2024-2025},
    	course={ELEC-H504 - Network Security},
    	author={SUNDARESAN Sankara\\ CHOUGULE Gaurav \\MESSAOUDI Leila\\ BOTTON David},
    	date={June 2025},
    	teacher={Pr. Jean-Michel Dricot\\ Navid Ladner},
    	logo={img/logo_vub_ulb.png},
    	manyAuthor,
    }
    
    % Switch to Computer Modern AFTER title page
    \renewcommand{\rmdefault}{cmr} % Computer Modern Roman
    \renewcommand{\sfdefault}{cmss} % Computer Modern Sans
    \renewcommand{\ttdefault}{cmtt} % Computer Modern Typewriter
    
    % Force font update (required after redefining defaults)
    \makeatletter
    \renewcommand{\reset@font}{\normalfont\@setfontsize\f@size{12}{14.4}}
    \makeatother
    \normalfont


    %
    % ABSTRACT 
    \begin{abstract}        
    This paper shows an operational deployment of the SSH honeypot using Cowrie on an Ubuntu EC2 instance to capture attacker activity under real-world circumstances. By exposing a knowingly open SSH port on the internet and securing legitimate access with a cryptographic key on a different port, the study observes and inspects adversary tactics, techniques, and procedures (TTPs). Key steps include isolating honeypot space from production access via fail2ban, redirecting malicious traffic to Cowrie via iptables, and customizing fake files to track attacker activity. All settings are covered to enable replication, as this paper remains focused on the hands-on aspect of honeypot-based deception. 
    \end{abstract}
    \newpage
    %
    % %


    %
    % section INTRODUCTION
    \section{Introduction}
    With more advanced automated SSH-based attacks, empirical analysis of attacker processes has been crucial in strengthening defenses. This report describes the configuration of a Cowrie honeypot in a test AWS environment that was designed to mimic realistic infrastructure but isolate malicious activity from legitimate administrative activity. Through routing port 22 traffic to the honeypot and restricting production access to port 2223 with key-based authentication, the study enables monitoring of attacker activities in fine granularity without compromising the security of systems. The fail2ban method is employed for counteracting brute-force attacks against hardened administrative interfaces, while rules based on iptables are implemented to route opponents into the honeypot infrastructure. 

        %
        % subsection PROBLEM STATEMENT
        \subsection{Problem Statement} 
        Public SSH services are most frequent targets of credential-stuffing attacks and post-compromise persistence methods such as SSH key injection and cronjob exploitation. Existing defensive measures have a tendency to lack the visibility to deeply analyze these malicious processes. This research addresses two significant challenges: (1) safe isolation of production access from honeypot bait to prevent collateral system compromise, and (2) the engineering of logging mechanisms that are able to capture attacker TTPs without affecting environmental integrity.
        %
        % %

        %
        % subsection RESEARCH QUESTIONS
        \subsection{Research Questions}
        The study investigates three core questions: First, which credential pairs are most frequently exploited in SSH brute-force campaigns against internet-exposed systems? Second, what persistence mechanisms (e.g., unauthorized key additions, scheduled task manipulation) do attackers prioritize following initial compromise? Third, to what extent can strategically placed decoy files, such as fabricated /etc/shadow entries, extend attacker engagement to improve TTP profiling?
        %
        % %
    %
    % %

        
    %
    % section SSH ISOLATION & FAIL2BAN HARDENING
    \section{SSH Isolation \& Fail2ban Hardening}

        
        \subsection{Admin SSH Hardening (Port 2223)}
        Restricts access to key-based auth for user \texttt{ubuntu}, reducing attack surface. Non-standard port avoids automated scans.
        
        \begin{lstlisting}[language=bash,caption={Securing Legitimate Access}]
 # /etc/ssh/sshd_config
 Port 2223
 Protocol 2
 HostKeyAlgorithms ssh-ed25519,rsa-sha2-512
 KexAlgorithms curve25519-sha256
 Ciphers chacha20-poly1305@openssh.com,aes256-gcm@openssh.com
 MACs hmac-sha2-512-etm@openssh.com
 PermitRootLogin no
 PasswordAuthentication no
 AllowUsers ubuntu
 LoginGraceTime 30s
 MaxAuthTries 2
 PubkeyAuthentication yes
 X11Forwarding no
        \end{lstlisting}
        
        
        \subsection{Fail2ban Configuration for Port 2223}
        Targets repeated public key failures, common in credential stuffing. Ignores password attempts (disabled in SSH).
        
        \begin{lstlisting}[language=bash,caption={Custom Jail Rules}]
 # /etc/fail2ban/jail.d/ssh-admin.conf
 [ssh-admin]
 enabled  = true
 port     = 2223
 filter   = sshd
 maxretry = 3  
 findtime = 10m
 bantime  = 30m
        \end{lstlisting}
        
        \begin{lstlisting}[language=bash,caption={Regex Filter Against Key-Based Attacks}]
 # /etc/fail2ban/filter.d/sshd.conf
failregex = %(cmnfailre)s
            <mdre-<mode>>
            ^%(__prefix_line)s(?: Received disconnect from <HOST> port \d+: Too many authentication failures | Disconnected from <HOST> port \d+ due to: Authentication failed for .* publickey )
            %(cfooterre)s
        \end{lstlisting}
        
        
        \subsection{IPTables Redirect for Cowrie (Port 22 $\rightarrow$ 2222)}
        Redirects all port 22 traffic to Cowrie while preserving admin access via 2223. Rules persist reboot. We do not whitelist admin IPs for ease of development.
        \begin{lstlisting}[language=bash,caption={Traffic Redirection Pre-Cowrie}]
 sudo iptables -t nat -A PREROUTING -p tcp --dport 22 -j REDIRECT --to-port 2222
 sudo ip6tables -t nat -A PREROUTING -p tcp --dport 22 -j REDIRECT --to-port 2222
 sudo netfilter-persistent save
        \end{lstlisting}
        
        %
        % %
    %
    % %


        \subsection{Validation Metrics}  
        \label{sec:validation}  
        
        To ensure rigorous validation, we verify component functionality and isolation using cybersecurity tools.  
            \hyperref[annexa:network]{Annex A} provides exact commands for:  Network isolation (nmap, iptables), Fail2ban regex precision (ssh-key brute-forcing), Cowrie logging integrity (attack replay), Attack surface audits (ss, netstat).
        









    %
    % %
    % citations go here \cite{Exemple}.
    
    %Bibliography
    % \nocite{*}
    % \printbibliography[type=article,title=Articles]
    



    
    %
    % % section ANNEXES
    \newpage
% \section*{Annexes}

\section{Annex: Validation for SSH Isolation \& Fail2ban Hardening}  
\label{annex:network}
\
\
\begin{lstlisting}[language=bash, label={annexes:network}, caption={Network Isolation Verification}]  
# Verify port 22 redirects to Cowrie (2222) and admin port (2223) is exclusive  
sudo nmap -sV -Pn -p 22,2222,2223 $EC2_PUBLIC_IP  

# Check iptables NAT rules for redirect (should show 22 to 2222)  
sudo iptables -t nat -L PREROUTING -v -n  

# Ensure no SSH service binds to port 22 (only Cowrie on 2222)  
sudo ss -tulpn | grep -E ':22|:2222|:2223'  
\end{lstlisting}  
\


\begin{lstlisting}[language=bash, label={annexes:fail2ban}, caption={Fail2ban Efficacy Testing}]  
# Simulate key-based brute-forcing to trigger fail2ban  
for i in {1..5}; do ssh -i ~/.ssh/wrong_key.pem ubuntu@localhost -p 2223; done  

# Verify fail2ban logged the bans (look for 'ssh-admin' jail)  
sudo grep "ssh-admin" /var/log/fail2ban.log  

# Check active bans (should list test IP)  
sudo fail2ban-client status ssh-admin  
\end{lstlisting}  
\


\begin{lstlisting}[language=bash, label={annexes:ssh-hardening}, caption={SSH Service Hardening Validation}]  
# 1. Verify active SSH configuration matches hardening intent (no fallback to weak protocols)  
sudo sshd -T | grep -E '^ciphers|^kexalgorithms|^macs|^hostkeyalgorithms'  

# 2. Test SSH service for protocol/cipher negotiation weaknesses  
nmap -Pn -p 2223 --script ssh2-enum-algos $EC2_PUBLIC_IP | grep -A 10 "algorithm negotiation"  

# 3. Confirm password authentication is globally disabled (even if bypass attempted)  
ssh -o PubkeyAuthentication=no -o PreferredAuthentications=password ubuntu@$EC2_PUBLIC_IP -p 2223  
\end{lstlisting}  
\





\newpage

\section{Annex: Cowrie Operational Validation}  
\label{annex:cowrie-validation}  
\
\
\begin{lstlisting}[language=bash,label={lst:cowrie-redirect},caption={Traffic Redirection Verification}]  
# 1. Validate iptables NAT rules  
sudo iptables -t nat -L PREROUTING -nv | grep 'tcp dpt:22 redir ports 2222'  

# Confirm no direct binding to port 22  
sudo nmap -sV -Pn -p 22,2222 $EC2_IP | grep -E '22/tcp|2222/tcp'  
\end{lstlisting}  
\
\begin{lstlisting}[language=bash,label={lst:cowrie-access},caption={Honeypot Engagement Testing}]  
# 2. Simulate attacker connection  
ssh -o StrictHostKeyChecking=no invalid_user@$EC2_IP -p 22  

# Verify session capture in logs  
sudo journalctl -u cowrie -f | grep 'SSH connection closed'  
\end{lstlisting}  
\
\begin{lstlisting}[language=bash,label={lst:cowrie-context},caption={Process Isolation Validation}]  
# 3. Confirm execution context  
ps -ef | grep cowrie | grep -v grep | awk '{print $1}' | uniq  
\end{lstlisting}  
\



\newpage

% NEW ANNEX HERE

\newpage 


\section{Annex: LLM Usage in this Project}  
\label{annex:llm}  
\
\
Large language models (LLMs) provided targeted support during this honeypot deployment, strictly limited to non-operational tasks. For documentation, LLMs assisted in drafting initial LaTeX templates for technical sections such as SSH hardening (2.1) and iptables redirection (2.3), with all configurations manually validated against AWS and Cowrie documentation. During troubleshooting, models proposed diagnostic commands for SSH permission conflicts (e.g., 'chmod' adjustments in §3.2), which were later tested in isolated Docker environments before EC2 implementation. LLMs were explicitly excluded from security-critical decisions: firewall rules, Fail2ban thresholds, and cryptographic settings in '/etc/ssh/sshd\_config' derived exclusively from NIST guidelines and Mozilla Infosec recommendations. No model influenced attacker engagement strategies, log analysis of captured TTPs, or live system interactions. 
All AI-generated content underwent peer review by the project’s cybersecurity team, with particular scrutiny applied to network redirection mechanics and user permission workflows. Final configurations reflect human expertise, with LLMs serving solely as productivity accelerators for non-sensitive administrative tasks.    

% END % Include external annex file  
    %
    % %


\end{document}

