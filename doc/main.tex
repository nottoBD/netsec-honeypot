%main.tex
\documentclass{cls/ULBreport}

% Fix section numbering (remove chapter prefix)
\renewcommand{\thesection}{\arabic{section}}
\renewcommand{\thesubsection}{\thesection.\arabic{subsection}}
\sceau{img/sceauULB.jpg}
\addbibresource{bib/biblio.bib}

% Keep fourier fonts for the title
\usepackage{ragged2e}
\usepackage{parskip}
\geometry{a4paper,top=1.5cm,bottom=2cm,left=2cm,right=2cm,headheight=15pt,includehead}

\usepackage{listings}
% For \AfterGroup hook
\usepackage{etoolbox}
% For framed boxes  
\usepackage[most]{tcolorbox}
% For colors  
\usepackage{xcolor}
\newcommand{\todo}[1]{\textcolor{red}{\textbf{TODO:} #1}}

% Configure abstract
\renewenvironment{abstract}{%
  \clearpage
  \thispagestyle{plain}
  \begin{center}
    \bfseries\itshape ABSTRACT
  \end{center}
  \vspace{0.5cm}
  \begin{quote}
    \itshape
    \justifying
}{%
  \end{quote}
  \clearpage
}

% Customize citation commands for footnotes
\DeclareFieldFormat{footnote:note}{#1}
\DeclareFieldFormat{footnote:shorttitle}{\mkbibemph{#1}}
\DeclareFieldFormat{footnote:author}{#1}
\DeclareFieldFormat{footnote:year}{\mkbibparens{#1}}
\DeclareFieldFormat{footnote:url}{\url{#1}}
% footnote spacing
\setlength{\footnotesep}{0.5em}

% Customization  bibliography
\DeclareFieldFormat{url}{\url{#1}}
\DeclareFieldFormat{note}{\textit{#1}}
\addbibresource{refs/references.bib}


% ========== Color Definitions ==========
\definecolor{background}{HTML}{FFFFFF}    % Pure white background
\definecolor{keyword}{HTML}{0000FF}       % Vivid blue for keywords
\definecolor{comment}{HTML}{008000}       % Rich green for comments
\definecolor{string}{HTML}{FF0000}        % Bright red for strings
\definecolor{identifier}{HTML}{000080}    % Deep navy for identifiers
\definecolor{number}{HTML}{800000}        % Maroon for numbers
\definecolor{frame}{HTML}{000000}         % Black frame

% ========== tcolorbox Style ==========
\newtcolorbox{codebox}{
    enhanced,
    arc=3pt,
    boxrule=1pt,
    colback=background,
    colframe=frame,
    top=4pt,
    bottom=4pt,
    left=4pt,
    right=4pt,
    overlay unbroken and first={
        \node[anchor=north east] at (frame.north east) {\lstlistingname};
    }
}

% ========== Listings Configuration ==========
\lstset{
    language=bash,
    basicstyle=\ttfamily\footnotesize,
    backgroundcolor=\color{background},
    keywordstyle=\color{keyword}\bfseries,
    commentstyle=\color{comment}\itshape,
    stringstyle=\color{string},
    identifierstyle=\color{identifier},
    numberstyle=\tiny\color{gray},
    numbers=none,
    frame=single,
    framerule=1pt,
    rulecolor=\color{frame},
    breaklines=true,
    breakatwhitespace=true, % Prevent breaks in middle of words
    postbreak=\mbox{\textcolor{red}{$\hookrightarrow$}\space},
    showstringspaces=false,
    tabsize=4,
    columns=fullflexible, % Better spacing for copy-paste
    keepspaces=true, % Preserve whitespace
    upquote=true, % Ensure proper quote rendering
    captionpos=b,
    belowcaptionskip=1em,
}

\begin{document}

    \titleULB{
    	title={Deception \& Honeypot for Attack Profiling},
    	studies={2024-2025},
    	course={ELEC-H504 - Network Security},
    	author={SUNDARESAN Sankara\\ CHOUGULE Gaurav \\MESSAOUDI Leila\\ BOTTON David},
    	date={June 2025},
    	teacher={Pr. Jean-Michel Dricot\\ Navid Ladner},
    	logo={img/logo_vub_ulb.png},
    	manyAuthor,
    }
    
    % Switch to Computer Modern AFTER title page
    \renewcommand{\rmdefault}{cmr} % Computer Modern Roman
    \renewcommand{\sfdefault}{cmss} % Computer Modern Sans
    \renewcommand{\ttdefault}{cmtt} % Computer Modern Typewriter
    
    % Force font update (required after redefining defaults)
    \makeatletter
    \renewcommand{\reset@font}{\normalfont\@setfontsize\f@size{12}{14.4}}
    \makeatother
    \normalfont


    %
    % ABSTRACT 
    \begin{abstract}        
    This paper shows an operational deployment of the \texttt{SSH} honeypot using \texttt{cowrie} on an \textit{Ubuntu} \texttt{EC2} instance to capture attacker activity in real-world circumstances. By exposing a knowingly open SSH port on the Internet and securing legitimate access with a cryptographic key on a different port, the study observes and inspects adversary tactics, techniques, and procedures (\textit{TTPs}). Key steps include isolating the honeypot space from production access using \texttt{fail2ban}, redirecting malicious traffic to \texttt{cowrie} via \texttt{iptables}, and forging artifacts to track attacker activity. A walkthrough of the settings is covered to demonstrate a complete implementation in \texttt{cowrie}, as this paper remains focused on the hands-on aspect of honeypot-based deception. This project's public \texttt{git} repository is available at \href{https://github.com/nottoBD/netsec-cowrie-honey}{that location}.
    \end{abstract}
    \newpage
    %
    % %


    %
    % section INTRODUCTION
    \section{Introduction}

As defined by \citet{spitzner2002honeypots}, 
\enquote{a honeypot is a security resource whose value lies in being probed, attacked, or compromised} (p. 58).

    
    Lance Spitzner, a Senior Instructor for SANS Cybersecurity Leadership, established foundational principles in his 2002 book \textit{Honeypots: Tracking Hackers}. 
Despite its age, Spitzner's core thesis retains striking relevance in modern threat intelligence; Honeypots derive value from \enquote{being probed, attacked, or compromised} (p. 23). Our Cowrie implementation on AWS positions itself onto that continuity, demonstrating that Spitzner's \enquote{gaining value from data} challenge (Ch.4) persists against contemporary attacks. Furthermore, Spitzner's risk mitigation framework (Ch.12) comprises the obstacle of signature-based detection tools that inform attackers about the nature of our operation, \enquote{a system designed to be attacked} (p. 298). Also, attacker behaviors documented in 2002 remain prevalent even today.
    With more advanced automated \texttt{SSH}-based attacks, empirical analysis of attacker processes has been crucial in strengthening defenses. By mimicking realistic infrastructure while isolating malicious activity from legitimate administrative access, this proof of concept applies Spitzner's principles of honeypot deployment, specifically leveraging \texttt{medium-interaction design} (Ch.5) to find the correct trade-off between risk containment and attacker engagement. Through silent redirection of open-to-the-public SSH traffic to the honeypot and restriction of legitimate access with key-based authentication, the study allows the monitoring of attacker activities in fine granularity without compromising the security of systems, proving Spitzner's assertion that honeypots provide \textit{«small amounts of high-value data»} without production noise.  

        %
        % subsection PROBLEM STATEMENT
        \subsection{Problem Statement} 
        Public \texttt{SSH} services are some of the most frequent targets of credential stuffing and post-compromise persistence attacks (e.g.,\texttt{SSH} key injection, cronjob exploitation). Existing defensive measures lack visibility of attacker's \texttt{TTPs} (Techniques, Tactics \& Procedures), a gap
        to deeply analyze these malicious processes. This research addresses two significant challenges: (1) safe isolation of production access from honeypot bait to prevent collateral system compromise, and (2) the engineering of logging mechanisms that are able to capture attacker TTPs without affecting environmental integrity.
        %
        % %

        %
        % subsection RESEARCH QUESTIONS
        \subsection{Research Questions}
        The report investigates three core questions: First, which credential pairs are most frequently exploited in SSH brute-force campaigns against internet-exposed systems? Second, what persistence mechanisms (e.g., unauthorized key additions, scheduled task manipulation) do attackers prioritize following initial compromise? Third, to what extent can strategically placed decoy files, such as fabricated /etc/shadow entries, extend attacker engagement to improve TTP profiling?
        %
        % %

    %
    % % DISCLAIMER
    \begin{tcolorbox}[  
        colback=red!5!white,  
        colframe=red!75!black,  
        title={\textbf{\textcolor{black}{Security Disclaimer}}},
        fontupper=\small,  
        sharp corners  
    ]  
    This setup only serves academic purposes; adversarial activities are welcomed, but no offensive counteraction shall be taken in return. Additionally, cloud provider configurations (e.g., IAM, networking) are not included as implementation-specific and beyond the scope of this paper.  
    \end{tcolorbox}  
    %
    % %

        
    %
    % section SSH ISOLATION & FAIL2BAN HARDENING
    \section{SSH Isolation \& Fail2ban Hardening}

        
        \subsection{Admin SSH Hardening}
        Restricts access to key-based auth on non-standard \textit{port 2223} for legitimate contributors to the project, thus reducing attack surface.
        
        \begin{lstlisting}[language=bash,caption={Securing Legitimate Access}]
 # /etc/ssh/sshd_config
 Port 2223
 Protocol 2
 HostKeyAlgorithms ssh-ed25519,rsa-sha2-512
 KexAlgorithms curve25519-sha256
 Ciphers chacha20-poly1305@openssh.com,aes256-gcm@openssh.com
 MACs hmac-sha2-512-etm@openssh.com
 PermitRootLogin no
 PasswordAuthentication no
 AllowUsers ubuntu
 LoginGraceTime 30s
 MaxAuthTries 2
 PubkeyAuthentication yes
 X11Forwarding no
        \end{lstlisting}
        
        
        \subsection{Fail2ban Configuration}
        Targets repeated public key failures, common in credential stuffing. Ignores password attempts (disabled in SSH).
        
        \begin{lstlisting}[language=bash,caption={Custom Jail Rules}]
 # /etc/fail2ban/jail.d/ssh-admin.conf
 [ssh-admin]
 enabled  = true
 port     = 2223
 filter   = sshd
 maxretry = 3  
 findtime = 10m
 bantime  = 30m
        \end{lstlisting}
        
        \begin{lstlisting}[language=bash,caption={Regex Filter Against Key-Based Attacks}]
 # /etc/fail2ban/filter.d/sshd.conf
failregex = %(cmnfailre)s
            <mdre-<mode>>
            ^%(__prefix_line)s(?: Received disconnect from <HOST> port \d+: Too many authentication failures | Disconnected from <HOST> port \d+ due to: Authentication failed for .* publickey )
            %(cfooterre)s
        \end{lstlisting}
        
        
        \subsection{IPtables Redirection}
        Redirects all \textit{port 22}'s traffic to \texttt{cowrie}, with rules persisting after reboot. Legitimate key-based accesses on \textit{port 2223} are separated and preserved, we could additionally whitelist our group collaborators' IP addresses for a much tighter defense but will not do it here for ease of development. The \texttt{sshd} to \texttt{cowrie} ({ports \textit{22}} $\rightarrow$ \textit{2222}) redirection is a security design with multiple benefits: we avoid both processes to conflict with each other, we can restart them separately, binding to high ports does not require root privileges, also, \texttt{fail2ban} needs an explicit target to be effective. Compartmentalization is a key principle for any deceptive operation, we also ensure an appropriate foundation for clean and comprehensive post-attack analysis. Find IPtables' official documentation \href{https://linux.die.net/man/8/iptables}{here}. 
        \begin{lstlisting}[language=bash,caption={Traffic Redirection to Cowrie}]
 sudo iptables -t nat -A PREROUTING -p tcp --dport 22 -j REDIRECT --to-port 2222
 sudo ip6tables -t nat -A PREROUTING -p tcp --dport 22 -j REDIRECT --to-port 2222
 sudo apt install iptables-persistent netfilter-persistent
 sudo netfilter-persistent save
        \end{lstlisting}
        
        %
        % %
    %
    % %


        \subsection{Validation Metrics}  
        \label{sec:validation}  
        
        To ensure rigorous validation, we verify component functionality and isolation using cybersecurity tools. The following \hyperref[annexes:network]{Annex} provides exact commands for:  Network isolation (nmap, \texttt{iptables}), \texttt{fail2ban} regex precision (ssh-key brute-forcing), \texttt{SSH} configuration enforcement and resilience regarding downgrade attacks, protocol validation and cryptographic compliance.
        
    \todo{Validate that these commands work once cowrie is running on 2222}


    \section{Cowrie Honeypot Deployment}
\label{sec:cowrie}
    \subsection{System Preparation}
    A specific non-root user and a Python virtual environment isolate Cowrie's operations and therefore ensure that a least-privilege account reduces lateral movement threats in case of compromise. Find the latest official documentation at \href{https://docs.cowrie.org/en/latest/INSTALL.html}{Cowrie Documentation}
    \begin{lstlisting}[language=bash,caption={Cowrie User Creation}]
sudo apt-get install -y git python3-venv python3-pip libssl-dev libffi-dev build-essential libpython3-dev authbind # to bind privileged ports without root priv
sudo adduser --disabled-password --gecos "" cowrie   # prevents password-based connections 
    \end{lstlisting}

    \subsection{Honeypot Configuration}
    Compartmentalize Cowrie within a virtual environment and enforce port isolation
    \begin{lstlisting}[language=bash,caption={Cowrie Honeypot Setup}]
# Operate as cowrie user
sudo su - cowrie  
git clone https://github.com/cowrie/cowrie  
cd cowrie  

# Isolate dependencies using Python venv
python3 -m venv cowrie-env  
source cowrie-env/bin/activate  
pip install --upgrade -r requirements.txt  

# Configure listener port (2222) to align with iptables redirection (section 2.3)
sed -i 's/tcp:6415:interface=127.0.0.1/tcp:2222:interface=0.0.0.0/' etc/cowrie.cfg  
    \end{lstlisting}


\subsection{Security Considerations}  
The rollout enforcing multiple layers of protection: Cowrie executes under a non-root privileged account with timed-out sudo privileges, implementing the least privilege principle. Attack surface minimization is offered by authbind on port binding, eliminating traditional root escalation threats by virtue of privileged port allocation. Compartmentalization is used for dependency isolation through Python virtual environments, preventing library conflicts and encapsulating potential exploits. Finally, fail2ban integration (Section~\ref{sec:fail2ban}) protects administratively selectively protects port 2223 with ongoing uncontrolled attacker interaction with the honeypot at port 2222. 

    %
    % %
    % citations go here \cite{Exemple}.
    
    %Bibliography
    \nocite{*}
    \printbibliography[type=article,title=Articles]








    
    %
    % %  section ANNEXES
    % horizontal line
    \addtocontents{toc}{%
      \protect\vspace{\baselineskip}%
      \protect\noindent\rule{\linewidth}{0.1pt}%
      \protect\par
    }
    % Change numbering to alphabetic
    \appendix
    \renewcommand{\thesection}{\Alph{section}}
    \setcounter{section}{0} % start at A
    \titleformat{\section}
      {\normalfont\Large\bfseries}
      {\Alph{section}}
      {1em}
      {}
    \newpage
% \section*{Annexes}

\section{Annex: Validation for SSH Isolation \& Fail2ban Hardening}  
\label{annex:network}
\
\
\begin{lstlisting}[language=bash, label={annexes:network}, caption={Network Isolation Verification}]  
# Verify port 22 redirects to Cowrie (2222) and admin port (2223) is exclusive  
sudo nmap -sV -Pn -p 22,2222,2223 $EC2_PUBLIC_IP  

# Check iptables NAT rules for redirect (should show 22 to 2222)  
sudo iptables -t nat -L PREROUTING -v -n  

# Ensure no SSH service binds to port 22 (only Cowrie on 2222)  
sudo ss -tulpn | grep -E ':22|:2222|:2223'  
\end{lstlisting}  
\


\begin{lstlisting}[language=bash, label={annexes:fail2ban}, caption={Fail2ban Efficacy Testing}]  
# Simulate key-based brute-forcing to trigger fail2ban  
for i in {1..5}; do ssh -i ~/.ssh/wrong_key.pem ubuntu@localhost -p 2223; done  

# Verify fail2ban logged the bans (look for 'ssh-admin' jail)  
sudo grep "ssh-admin" /var/log/fail2ban.log  

# Check active bans (should list test IP)  
sudo fail2ban-client status ssh-admin  
\end{lstlisting}  
\


\begin{lstlisting}[language=bash, label={annexes:ssh-hardening}, caption={SSH Service Hardening Validation}]  
# 1. Verify active SSH configuration matches hardening intent (no fallback to weak protocols)  
sudo sshd -T | grep -E '^ciphers|^kexalgorithms|^macs|^hostkeyalgorithms'  

# 2. Test SSH service for protocol/cipher negotiation weaknesses  
nmap -Pn -p 2223 --script ssh2-enum-algos $EC2_PUBLIC_IP | grep -A 10 "algorithm negotiation"  

# 3. Confirm password authentication is globally disabled (even if bypass attempted)  
ssh -o PubkeyAuthentication=no -o PreferredAuthentications=password ubuntu@$EC2_PUBLIC_IP -p 2223  
\end{lstlisting}  
\





\newpage

\section{Annex: Cowrie Operational Validation}  
\label{annex:cowrie-validation}  
\
\
\begin{lstlisting}[language=bash,label={lst:cowrie-redirect},caption={Traffic Redirection Verification}]  
# 1. Validate iptables NAT rules  
sudo iptables -t nat -L PREROUTING -nv | grep 'tcp dpt:22 redir ports 2222'  

# Confirm no direct binding to port 22  
sudo nmap -sV -Pn -p 22,2222 $EC2_IP | grep -E '22/tcp|2222/tcp'  
\end{lstlisting}  
\
\begin{lstlisting}[language=bash,label={lst:cowrie-access},caption={Honeypot Engagement Testing}]  
# 2. Simulate attacker connection  
ssh -o StrictHostKeyChecking=no invalid_user@$EC2_IP -p 22  

# Verify session capture in logs  
sudo journalctl -u cowrie -f | grep 'SSH connection closed'  
\end{lstlisting}  
\
\begin{lstlisting}[language=bash,label={lst:cowrie-context},caption={Process Isolation Validation}]  
# 3. Confirm execution context  
ps -ef | grep cowrie | grep -v grep | awk '{print $1}' | uniq  
\end{lstlisting}  
\



\newpage

% NEW ANNEX HERE

\newpage 


\section{Annex: LLM Usage in this Project}  
\label{annex:llm}  
\
\
Large language models (LLMs) provided targeted support during this honeypot deployment, strictly limited to non-operational tasks. For documentation, LLMs assisted in drafting initial LaTeX templates for technical sections such as SSH hardening (2.1) and iptables redirection (2.3), with all configurations manually validated against AWS and Cowrie documentation. During troubleshooting, models proposed diagnostic commands for SSH permission conflicts (e.g., 'chmod' adjustments in §3.2), which were later tested in isolated Docker environments before EC2 implementation. LLMs were explicitly excluded from security-critical decisions: firewall rules, Fail2ban thresholds, and cryptographic settings in '/etc/ssh/sshd\_config' derived exclusively from NIST guidelines and Mozilla Infosec recommendations. No model influenced attacker engagement strategies, log analysis of captured TTPs, or live system interactions. 
All AI-generated content underwent peer review by the project’s cybersecurity team, with particular scrutiny applied to network redirection mechanics and user permission workflows. Final configurations reflect human expertise, with LLMs serving solely as productivity accelerators for non-sensitive administrative tasks.    

% END
    %
    % %




\end{document}
